

\section{Control FPGA Programming Model}
\label{section_1}

%
\begin {table}[H]
\begin{center}
\caption {Control FPGA addresses offsets}
\label{tab_top_reg}
\begin{tabular}{|l|l|l|l|l|}
\hline
Offset (hex)& Type & Name & Size(bytes) & Comments \\
\hline
0x00000001 & R&Firmware\_Version& 4 & \\
\hline
0x00000002 & R &General\_Status& 4 & \\
\hline
0x00000004 & RW &General\_Control& 4 & \\
\hline
0x00000008 & RW&TTC\_Forwarding\_Control& 4 & \\
\hline
0x00000010 & RW&I2C\_Data\_Out& 4 & \\
\hline
0x00000011 & R&I2C\_Error& 4 & \\
\hline
0x00000012 & RW&I2C\_Address& 4 & \\
\hline
0x00000013 & RW&I2C\_WriteEnable& 4 & \\
\hline
0x00000014 & RW&I2C\_Pointer& 4 & \\
\hline
0x00000015 & RW&I2C\_Data\_In& 4 & \\
\hline
0x00000016 & RW&I2C\_Bus\_Select& 4 & \\
\hline
0x00000020 & R&SGMII\_Phy\_Error\_Counter& 4&  \\
\hline
0x00000021 & R&Ethernet\_MAC\_Error\_Counter& 4 & \\
\hline
0x00000022 & R&CTRL\_BUS\_U1\_ErrorCounter& 4 & \\
\hline
0x00000023 & R&CTRL\_BUS\_U2\_Error\_Counter& 4 & \\
\hline
0x00000024 & R&I2C\_Error\_Counter& 4 & \\
\hline
0x00000100 & RW&ROD\_Infrastructure& 128 & sub-space offset \\
\hline
0x00001000 & RW&Test\_RAM& 1024 & sub-space offset \\
\hline
\end{tabular}
\end{center}
\end{table}
%



\subsection{Firmware Version}

\begin {table}[H]
\begin{center}
\caption {Firmware Version}
\label{reg_4}
\begin{tabular}{|l|l|}
\hline
\textbf{bit} & \textbf{Function} \\
\hline
0-2 & FW version \\
\hline
2-31& Unused  \\
\hline
\end{tabular}
\end{center}
\end{table}



\subsection{General\_Control}
General control register of the ControlFPGA.

\begin {table}[H]
\begin{center}
\caption {Firmware Version}
\label{reg_4}
\begin{tabular}{|l|l|}
\hline
\textbf{bit} & \textbf{Function} \\
\hline
0-5 & ?? \\
\hline
5-31& Unused  \\
\hline
\end{tabular}
\end{center}
\end{table}
%
\begin{itemize}
\item \textbf{Bit 0}: ...
\item \textbf{Bit 1-2}: ...
\item \textbf{Bit 2-32}: unused
\end{itemize}


\subsection{SGMII\_Phy\_Error\_Counter}
ErrorCounter for the SGMII Phy (errors in the 8b10b encoded data).
%
%\begin{figure}[H]
%    \centering
%    \includegraphics[width = 0.45\textheight]{/afs/cern.ch/user/e/esimioni/workspace/Talks/images/IPbus_errors.png}
%    \caption{IPBus error counters}
%    \label{IPB_err}
%\end{figure}
%
\begin {table}[H]
\begin{center}
\caption {Firmware Version}
\label{reg_4}
\begin{tabular}{|l|l|}
\hline
\textbf{bit} & \textbf{Function} \\
\hline
0-7 & SGMII error counter\\
\hline
8-15 & MAC error counter \\
\hline
16-32 & unused \\
\end{tabular}
\end{center}
\end{table}
%
\begin{itemize}
\item \textbf{Bit 0}:...
\end{itemize}
%
\subsection{TTC\_Forwarding\_Control}
Control register for the TTC forwarding.

\subsection{I2C}
The various I2C registers are here described.

\subsection{Ethernet\_MAC\_Error\_Counter}
ErrorCounter for the Ethernet MAC (unsuccessful frame reception -> assertion of the rx\_axis\_mac\_tuser signal). Errors in the SGMII Phy will also lead to errors in the MAC.

\subsection{CTRL\_BUS\_U1\_ErrorCounter}
ErrorCounter for the incoming IPbus lines from the U1 processor (errors in the 8b10b encoded data).

\subsection{CTRL\_BUS\_U2\_Error\_Counter}


\subsection{ROD\_Infrastructure}
%
\begin {table}[H]
\begin{center}
\caption {ROD control and status registers for Run Control}
\label{rod_control_run}
\begin{tabular}{|l|l|l|l|}
\hline
Offset (hex)& Type & Register Name & Size(bytes)\\
\hline
0x00000000 & RW & Ov\_Busy & 4 \\
\hline
0x00000001 & RW & Gen\_L1A & 4 \\
\hline
0x00000002 & RW & Run\_Type\_Nbr & 4 \\
\hline
0x00000003 & RW & Trg\_Type & 4 \\
\hline
0x00000004 & RW & Level1ID\_Gen & 4 \\
\hline
0x00000005 & W & Run\_Reset & 4 \\
\hline
0x00000006 & R & ROD\_Sys\_Fw\_Ver & 4 \\
\hline
0x00000007-0A & R & Run\_Gen\_Dbg & 16 \\
\hline
0x0000000A-0F & RW & Run\_Reserved & unused \\
\hline
\end{tabular}
\end{center}
\end{table}


\begin {table}[H]
\begin{center}
\caption {ROD control and status registers for ROD Control}
\label{rod_control_rod}
\begin{tabular}{|l|l|l|l|}
\hline
Offset (hex)& Type & Register Name & Size(bytes)\\
\hline
0x00000010 & W & ROD\_Reset & 4 \\
\hline
0x00000011-12 & RW & Fifo\_Thr & 8 \\
\hline
0x00000013 & RW & Busy\_Idle\_Fr\_Conf & 4 \\
\hline
0x00000014-18 & RW & Hist\_Conf & 20 \\
\hline
0x00000019-02D & R & Err\_Ctr & 16 \\
\hline
0x0000002E-031 & R & ROD\_Gen\_Dbg & 16 \\
\hline
0x00000032 & R & Busy\_Idle\_Fr & 4 \\
\hline
0x00000033-034 & R & ROD\_Hist & 8 \\
\hline
0x00000035-03E & R & ROD\_Fifo\_Stat & 40 \\
\hline
0x0000003F & RW & ROD\_Reserved & unused \\
\hline
\end{tabular}
\end{center}
\end{table}


\begin {table}[H]
\begin{center}
\caption {ROD control and status registers for DDR-GTX Control}
\label{rod_control_ddrgtx}
\begin{tabular}{|l|l|l|l|}
\hline
Offset (hex)& Type & Register Name & Size(bytes)\\
\hline
0x00000040 & W & Link\_Reset & 4 \\
\hline
0x00000041 & RW & Link\_Enable & 4 \\
\hline
0x00000042-43 & RW & IDelays & 8 \\
\hline
0x00000044 & R & Syn\_Status & 4 \\
\hline
0x00000045-4D & R & Byte\_Ctr & 32 \\
\hline
0x0000004E-056 & R & Err\_Ctr & 32 \\
\hline
0x00000057-5B & R & DDR\_Gen\_Dbg & 16 \\
\hline
0x0000005B-5F & RW & DDR\_Reserved & unused \\
\hline
\end{tabular}
\end{center}
\end{table}


\begin {table}[H]
\begin{center}
\caption {ROD control and status registers for S-Link}
\label{rod_control_slink}
\begin{tabular}{|l|l|l|l|}
\hline
Offset (hex)& Type & Register Name & Size(bytes)\\
\hline
0x00000030 & W & Slink\_Reset & 4 \\
\hline
0x00000031 & RW & Slink\_Enable & 4 \\
\hline
0x00000032 & RW & Format\_ROS\_Ver & 4 \\
\hline
0x00000033 & RW & Format\_ROIB\_Ver & 4 \\
\hline
0x00000034 & RW & SubDet\_Module\_ID & 4 \\
\hline
0x00000035 & RW & Busy\_Idle\_Fr\_Conf & 4 \\
\hline
0x00000036 & R & Slink\_Status & 4 \\
\hline
0x00000037-38 & R & Busy\_Idle\_Fr & 4 \\
\hline
0x00000038-3F & RW & Slink\_Reserved & unused \\
\hline
\end{tabular}
\end{center}
\end{table}



\subsubsection{Ov\_Busy}

\begin {table}[H]
\begin{center}
\begin{tabular}{|l|l|}
\hline
\textbf{bit} & \textbf{Function} \\
\hline
0-2 & ?? \\
\hline
3-16 & ??  \\
\hline
17-31 & unused \\
\end{tabular}
\end{center}
\end{table}


\begin{itemize}
\item DIAGNOSTICS REGISTER: Enables overwriting existing BUSY status in certain places of the system.
\end{itemize}



\subsubsection{Gen\_L1A}

\begin {table}[H]
\begin{center}
\begin{tabular}{|l|l|}
\hline
\textbf{bit} & \textbf{Function} \\
\hline
0 & Enable internal L1A generation \\
\hline
1 & 0 - random, 1 - fixed frequency  \\
\hline
2-3 & unused \\
\hline
4-31 & frequency (at sysclk counts) \\
\hline
\end{tabular}
\end{center}
\end{table}


\begin{itemize}
\item DIAGNOSTICS REGISTER: Enables the internal L1A signal generation at a given frequency
\end{itemize}



\subsubsection{Run\_Type\_Nbr}

\begin {table}[H]
\begin{center}
\begin{tabular}{|l|l|}
\hline
\textbf{bit} & \textbf{Function} \\
\hline
0-23 & Run Number \\
\hline
24-31 & Run Type  \\
\hline
\end{tabular}
\end{center}
\end{table}


\begin{itemize}
\item Sets the Run Number and Run Type fields for Slink packet headers
\end{itemize}



\subsubsection{Trg\_Type}

\begin {table}[H]
\begin{center}
\begin{tabular}{|l|l|}
\hline
\textbf{bit} & \textbf{Function} \\
\hline
0 & 0 - Trigger Type from TTC, 1 - Trigger Type from IPBus \\
\hline
1-7 & Reserved  \\
\hline
8-15 & Trigger Type value \\
\hline
\end{tabular}
\end{center}
\end{table}


\begin{itemize}
\item Enables the trigger type to be provided via slow control with a given value
\end{itemize}



\subsubsection{Level1ID\_Gen}

\begin {table}[H]
\begin{center}
\begin{tabular}{|l|l|}
\hline
\textbf{bit} & \textbf{Function} \\
\hline
0 & 0 - L1AID from TTC, 1 - L1AID from internal counter \\
\hline
1 & 0 - Fixed internal value, 1 - Incremented \\
\hline
2-7 & Reserved \\
\hline
8-31 & L1AID initial value for internal counting  \\
\hline
\end{tabular}
\end{center}
\end{table}


\begin{itemize}
\item Enables the trigger type to be provided via slow control with a given value
\end{itemize}




\subsubsection{Run\_Reset}

\begin {table}[H]
\begin{center}
\begin{tabular}{|l|l|}
\hline
\textbf{bit} & \textbf{Function} \\
\hline
0 & Reset Pulse \\
\hline
1-31 & Reserved  \\
\hline
\end{tabular}
\end{center}
\end{table}


\begin{itemize}
\item Resets all the ROD infrastructure components (ROD, DDRs, Slink)
\end{itemize}



\subsubsection{ROD\_Sys\_Fw\_Ver}

\begin {table}[H]
\begin{center}
\begin{tabular}{|l|l|}
\hline
\textbf{bit} & \textbf{Function} \\
\hline
0-31 & Firmware version number \\
\hline
\end{tabular}
\end{center}
\end{table}


\begin{itemize}
\item Firmware version number for the ROD infrastructure
\end{itemize}




\subsubsection{Run\_Gen\_Dbg}

\begin {table}[H]
\begin{center}
\begin{tabular}{|l|l|}
\hline
\textbf{bit} & \textbf{Function} \\
\hline
0-31 & Debug word 0 \\
\hline
32-63 & Debug word 1 \\
\hline
64-95 & Debug word 2 \\
\hline
96-127 & Debug word 3 \\
\hline
\end{tabular}
\end{center}
\end{table}

\begin{itemize}
\item DIAGNOSTICS REGISTER: General debugging words for the ROD infrastructure
\end{itemize}

%%%%%%%%%%%%%%%%%%%%%%5
%ROD ctrl and status
%%%%%%%%%%%%%%%%%%%%%%5


%%%%%
\subsubsection{ROD\_Reset}
\begin {table}[H]
\begin{center}
\begin{tabular}{|l|l|}
\hline
\textbf{bit} & \textbf{Function} \\
\hline
0 & Resets the ROD control, cleans all internal buffers. \\
\hline
1-31 & Reserved \\
\hline
\end{tabular}
\caption{Resets only ROD part.}
\end{center}
\end{table}

%%%%%
\subsubsection{ Fifo\_Thr}
\begin {table}[H]
\begin{center}
\begin{tabular}{|l|l|}
\hline
\textbf{bit} & \textbf{Function} \\
\hline
0-15 & fill level of internal data buffers, above which busy is sent\\
\hline
16-31 & ammount of L1A in queue awaiting for processing above which busy is sent \\
\hline
63-32 & reserved \\
\hline
\end{tabular}
\caption{Busy conditioning based on FIFO fill levels.}
\end{center}
\end{table}

%%%%%
\subsubsection{Busy\_Idle\_Fr\_Conf}
\begin {table}[H]
\begin{center}
\begin{tabular}{|l|l|}
\hline
\textbf{bit} & \textbf{Function} \\
\hline
0-31 & time in which a percentage of ROD busy is calculated - value mltiplied by 25ns\\
\hline
\end{tabular}
\caption{Control of fraction register}
\end{center}
\end{table}

%%%%%
\subsubsection{Hist\_Conf}
\begin {table}[H]
\begin{center}
\begin{tabular}{|l|l|}
\hline
\textbf{bit} & \textbf{Function} \\
\hline
0-7 & multiplexer for 1st histogram (TBD)\\ \hline
8-15 & multiplexer for 2nd histogram\\ \hline
16-23 & multiplexer for 3rd histogram\\ \hline
24-31 & multiplexer for 4th histogram\\ \hline
32-47 & time when one bin of histogram is accumulated (1st) - value*25ns \\ \hline
48-63 & time when one bin of histogram is accumulated (2nd) - value*25ns \\ \hline
64-79 & time when one bin of histogram is accumulated (3rd) - value*25ns \\ \hline
80-95 & time when one bin of histogram is accumulated (4th) - value*25ns \\ \hline
96-111 & threshold above which value of current bin of histogram is increased (1st)\\ \hline
112-127 & threshold above which value of current bin of histogram is increased (2nd)\\ \hline
128-143 & threshold above which value of current bin of histogram is increased (3rd)\\ \hline
144-160 & threshold above which value of current bin of histogram is increased (4th)\\ \hline
\end{tabular}
\caption{Histogramming of incoming data - configurable real time statistics. For example TOB number in function of time.}
\end{center}
\end{table}

%%%%%
\subsubsection{Err\_Ctr}
\begin {table}[H]
\begin{center}
\begin{tabular}{|l|l|}
\hline
\textbf{bit} & \textbf{Function} \\
\hline
0-7 & CRC error cntr for input 0 \\ \hline
8-15 & CRC error cntr for input 1 \\ \hline
... & ... \\ \hline
632-639 & CRC error cntr for input 79 \\ \hline
\hline
\end{tabular}
\caption{CRC error couters.}
\end{center}
\end{table}

%%%%%
\subsubsection{ROD\_Gen\_Dbg}
\begin {table}[H]
\begin{center}
\begin{tabular}{|l|l|}
\hline
\textbf{bit} & \textbf{Function} \\
\hline
0-127 & TBD \\
\hline
\end{tabular}
\caption{Internal state machines, statistics, debug information.}
\end{center}
\end{table}

%%%%%
\subsubsection{Busy\_Idle\_Fr}
\begin {table}[H]
\begin{center}
\begin{tabular}{|l|l|}
\hline
\textbf{bit} & \textbf{Function} \\
\hline
0-31 & time when ROD busy was set (value * 25ns)  \\
\hline
\end{tabular}
\caption{Shows busy time fraction, can be used together with s-link busy for debug or dignostics.}
\end{center}
\end{table}

%%%%%
\subsubsection{ROD\_Hist}
\begin {table}[H]
\begin{center}
\begin{tabular}{|l|l|}
\hline
\textbf{bit} & \textbf{Function} \\
\hline
0-15 & data read out from 1st histogram, 512 bins\\ \hline
16-31 & data read out from 2nd histogram, 512 bins\\ \hline
32-47 & data read out from 3rd histogram, 512 bins\\ \hline
48-63 & data read out from 4th histogram, 512 bins\\ \hline
\end{tabular}
\caption{Histogrammed data - when reading this address user reads out histogram saved in FIFO. Only when all data is read out ($16^2$ 32bit data words) next histogram will be created.}
\end{center}
\end{table}

%%%%%
\subsubsection{ROD\_Fifo\_Stat}
\begin {table}[H]
\begin{center}
\begin{tabular}{|l|l|}
\hline
\textbf{bit} & \textbf{Function} \\
\hline
0-1 & link 0 FIFO empty(0), full(1) \\ \hline
2-3 & link 1 FIFO empty(2), full(3) \\ \hline
... & ...\\ \hline
178-179 & link 79 FIFO empty(178), full(179)\\ \hline
180-188 & L1A FIFO level\\ \hline
\end{tabular}
\caption{Empty and Full for link fifos and l1A FIFO fill level.}
\end{center}
\end{table}

%%%%%%%%%%%%%%%%%%%%%%5
%DDR/GT
%%%%%%%%%%%%%%%%%%%%%%5


\subsubsection{Link\_Reset}

\begin {table}[H]
\begin{center}
\begin{tabular}{|l|l|}
\hline
\textbf{bit} & \textbf{Function} \\
\hline
0 & ROD DDR Reset Pulse \\
\hline
1-31 & Reserved \\
\hline
\end{tabular}
\end{center}
\end{table}


\begin{itemize}
\item Resets the ROD DDR links
\end{itemize}




\subsubsection{Link\_Enable}

\begin {table}[H]
\begin{center}
\begin{tabular}{|l|l|}
\hline
\textbf{bit} & \textbf{Function} \\
\hline
0 & DDR Link 0 \\
\hline
1 & DDR Link 1 \\
\hline
2 & DDR Link 2 \\
\hline
3 & DDR Link 3 \\
\hline
4 & DDR Link 4 \\
\hline
5 & DDR Link 5 \\
\hline
6 & DDR Link 6 \\
\hline
7 & DDR Link 7 \\
\hline
8 & DDR Link 8 \\
\hline
9 & DDR Link 9 \\
\hline
10 & DDR Link 10 \\
\hline
11 & DDR Link 11 \\
\hline
12 & DDR Link 12 \\
\hline
13-31 & Reserved \\
\hline
\end{tabular}
\end{center}
\end{table}


\begin{itemize}
\item Enables or disables a given ROD DDR Link (0 - disable, 1 - enable)
\end{itemize}




\subsubsection{IDelays}

\begin {table}[H]
\begin{center}
\begin{tabular}{|l|l|}
\hline
\textbf{bit} & \textbf{Function} \\
\hline
0-4 & Delay value for Link 0 \\
\hline
5-9 & Delay value for Link 1 \\
\hline
10-14 & Delay value for Link 2 \\
\hline
15-19 & Delay value for Link 3 \\
\hline
20-24 & Delay value for Link 4 \\
\hline
25-29 & Delay value for Link 5 \\
\hline
30-34 & Delay value for Link 6 \\
\hline
35-39 & Delay value for Link 7 \\
\hline
40-44 & Delay value for Link 8 \\
\hline
45-49 & Delay value for Link 9 \\
\hline
50-54 & Delay value for Link 10 \\
\hline
55-59 & Delay value for Link 11 \\
\hline
60-64 & Delay value for Link 12 \\
\hline
65-95 & Reserved \\
\hline
\end{tabular}
\end{center}
\end{table}


\begin{itemize}
\item DIAGNOSTICS REGISTER: IDelay values for individual ROD DDR links
\end{itemize}



\subsubsection{Syn\_Status}

\begin {table}[H]
\begin{center}
\begin{tabular}{|l|l|}
\hline
\textbf{bit} & \textbf{Function} \\
\hline
0 & Synchronization status for Link 0 \\
\hline
1 & Synchronization status for Link 1 \\
\hline
2 & Synchronization status for Link 2 \\
\hline
3 & Synchronization status for Link 3 \\
\hline
4 & Synchronization status for Link 4 \\
\hline
5 & Synchronization status for Link 5 \\
\hline
6 & Synchronization status for Link 6 \\
\hline
7 & Synchronization status for Link 7 \\
\hline
8 & Synchronization status for Link 8 \\
\hline
9 & Synchronization status for Link 9 \\
\hline
10 & Synchronization status for Link 10 \\
\hline
11 & Synchronization status for Link 11 \\
\hline
12 & Synchronization status for Link 12 \\
\hline
13-31 & Reserved \\
\hline
\end{tabular}
\end{center}
\end{table}


\begin{itemize}
\item DIAGNOSTICS REGISTER: Synchronization status for individual ROD DDR links
\end{itemize}




\subsubsection{Byte\_Ctr}

\begin {table}[H]
\begin{center}
\begin{tabular}{|l|l|}
\hline
\textbf{bit} & \textbf{Function} \\
\hline
0-15 & Bytes counter for Link 0 \\
\hline
15-31 & Bytes counter for Link 1 \\
\hline
32-47 & Bytes counter for Link 2 \\
\hline
48-63 & Bytes counter for Link 3 \\
\hline
64-79 & Bytes counter for Link 4 \\
\hline
80-95 & Bytes counter for Link 5 \\
\hline
96-111 & Bytes counter for Link 6 \\
\hline
112-123 & Bytes counter for Link 7 \\
\hline
\end{tabular}
\end{center}
\end{table}


\begin{itemize}
\item DIAGNOSTICS REGISTER: Transmitted bytes counters for individual ROD DDR links
\end{itemize}



\subsubsection{Err\_Ctr}

\begin {table}[H]
\begin{center}
\begin{tabular}{|l|l|}
\hline
\textbf{bit} & \textbf{Function} \\
\hline
0-15 & Error counter for Link 0 \\
\hline
15-31 & Error counter for Link 1 \\
\hline
32-47 & Error counter for Link 2 \\
\hline
48-63 & Error counter for Link 3 \\
\hline
64-79 & Error counter for Link 4 \\
\hline
80-95 & Error counter for Link 5 \\
\hline
96-111 & Error counter for Link 6 \\
\hline
112-123 & Error counter for Link 7 \\
\hline
\end{tabular}
\end{center}
\end{table}


\begin{itemize}
\item Invalid characters received counters for individual ROD DDR links
\end{itemize}



\subsubsection{DDR\_Gen\_Dbg}

\begin {table}[H]
\begin{center}
\begin{tabular}{|l|l|}
\hline
\textbf{bit} & \textbf{Function} \\
\hline
0-31 & Debug word 0 \\
\hline
32-63 & Debug word 1 \\
\hline
64-95 & Debug word 2 \\
\hline
96-127 & Debug word 3 \\
\hline
\end{tabular}
\end{center}
\end{table}


\begin{itemize}
\item DIAGNOSTICS REGISTER: General debugging words for the ROD DDR
\end{itemize}




\subsubsection{Slink\_Reset}

\begin {table}[H]
\begin{center}
\begin{tabular}{|l|l|}
\hline
\textbf{bit} & \textbf{Function} \\
\hline
0 & Slink to ROS Reset Pulse \\
\hline
1 & Slink to ROIB Reset Pulse \\
\hline
2-31 & Reserved \\
\hline
\end{tabular}
\end{center}
\end{table}


\begin{itemize}
\item Resets the individual Slink connections
\end{itemize}



\subsubsection{Slink\_Enable}

\begin {table}[H]
\begin{center}
\begin{tabular}{|l|l|}
\hline
\textbf{bit} & \textbf{Function} \\
\hline
0 & Slink to ROS Enable \\
\hline
1 & Slink to ROIB Enable \\
\hline
2-31 & Reserved \\
\hline
\end{tabular}
\end{center}
\end{table}


\begin{itemize}
\item Enables or disables a given SLink connection (0 - disable, 1 - enable)
\end{itemize}



\subsubsection{Format\_ROS\_Ver}

\begin {table}[H]
\begin{center}
\begin{tabular}{|l|l|}
\hline
\textbf{bit} & \textbf{Function} \\
\hline
0-15 & Minor Format Version \\
\hline
16-31 & Major Format Version \\
\hline
\end{tabular}
\end{center}
\end{table}


\begin{itemize}
\item Sets the format versions for the SLink connection to ROS
\end{itemize}



\subsubsection{Format\_ROIB\_Ver}

\begin {table}[H]
\begin{center}
\begin{tabular}{|l|l|}
\hline
\textbf{bit} & \textbf{Function} \\
\hline
0-15 & Minor Format Version \\
\hline
16-31 & Major Format Version \\
\hline
\end{tabular}
\end{center}
\end{table}


\begin{itemize}
\item Sets the format versions for the SLink connection to ROIB
\end{itemize}




\subsubsection{SubDet\_Module\_ID}

\begin {table}[H]
\begin{center}
\begin{tabular}{|l|l|}
\hline
\textbf{bit} & \textbf{Function} \\
\hline
0-7 & Subdetector ID \\
\hline
8-15 & Reserved \\
\hline
16-31 & Module ID \\
\hline
\end{tabular}
\end{center}
\end{table}


\begin{itemize}
\item Sets the Subdetector ID and Module ID for the SLink connections
\end{itemize}




\subsubsection{Busy\_Idle\_Fr\_Conf}

\begin {table}[H]
\begin{center}
\begin{tabular}{|l|l|}
\hline
\textbf{bit} & \textbf{Function} \\
\hline
0-31 & Time period \\
\hline
\end{tabular}
\end{center}
\end{table}


\begin{itemize}
\item DIAGNOSTICS REGISTER: Sets the time period for the calculation of Busy and Idle fractions of Slink connections (at sysclk ticks)
\end{itemize}



\subsubsection{Slink\_Status}

\begin {table}[H]
\begin{center}
\begin{tabular}{|l|l|}
\hline
\textbf{bit} & \textbf{Function} \\
\hline
0 & ROS Slink down \\
\hline
1 & ROS Slink fifo full \\
\hline
2 & ROS Slink link present \\
\hline
3 & Reserved \\
\hline
4 & ROIB Slink down \\
\hline
5 & ROIB Slink fifo full \\
\hline
6 & ROIB Slink link present \\
\hline
7-31 & Reserved \\
\hline
\end{tabular}
\end{center}
\end{table}


\begin{itemize}
\item Shows the status of the Slink connections to the subsystems
\end{itemize}




\subsubsection{Busy\_Idle\_Fr}

\begin {table}[H]
\begin{center}
\begin{tabular}{|l|l|}
\hline
\textbf{bit} & \textbf{Function} \\
\hline
0-31 & ROS Slink Busy time Fraction \\
\hline
32-63 & ROIB Slink Busy time Fraction \\
\hline
\end{tabular}
\end{center}
\end{table}


\begin{itemize}
\item DIAGNOSTICS REGISTER: Shows the activity fraction of the Slink connections to the subsystems
\end{itemize}








\subsection{ROD\_Processor\_Infrastructure}
%
\begin {table}[H]
\begin{center}
\caption {ROD control and status registers for Run Control}
\label{rod_control_run}
\begin{tabular}{|l|l|l|l|}
\hline
Offset (hex)& Type & Register Name & Size(bytes)\\
\hline
0x00000000 & RW & Ov\_Busy & 4 \\
\hline
0x00000001 & RW & ROD\_Fw\_Ver & 4 \\
\hline
0x00000002-03 & RW & ROD\_Dbg & 8 \\
\hline
0x00000004-0F & RW & Run\_Reserved & unused \\
\hline
\end{tabular}
\end{center}
\end{table}


\begin {table}[H]
\begin{center}
\caption {ROD control and status registers for ROD Control}
\label{rod_control_run}
\begin{tabular}{|l|l|l|l|}
\hline
Offset (hex)& Type & Register Name & Size(bytes)\\
\hline
0x00000000 & W & ROD\_Reset & 4 \\
\hline
0x00000001-0A & RW & ROD\_Slices & 40 \\
\hline
0x0000000B-14 & RW & ROD\_Offsets & 40 \\
\hline
0x000000015 & RW & ROD\_L1A\_Offset & 4 \\
\hline
0x000000016 & R & ROD\_L1A\_Fill & 4 \\
\hline
0x000000017-18 & RW & ROD\_Dbg & 8 \\
\hline
\end{tabular}
\end{center}
\end{table}


%%%%%
\subsubsection{ROD\_Reset}
\begin {table}[H]
\begin{center}
\begin{tabular}{|l|l|}
\hline
\textbf{bit} & \textbf{Function} \\
\hline
0 & Reset ROD logic and clean buffers \\ \hline
1-31 & Reserved \\ \hline
... & ...\\ \hline
178-179 & link 79 FIFO empty(178), full(179)\\ \hline
180-188 & L1A FIFO level\\ \hline
\end{tabular}
\caption{Bit 0 of this register is resets ROD logic on processor side (pulse).}
\end{center}
\end{table}

%%%%%
\subsubsection{ROD\_Slices}
\begin {table}[H]
\begin{center}
\begin{tabular}{|l|l|}
\hline
\textbf{bit} & \textbf{Function} \\
\hline
0-2 & Number of slices accepted after L1A for link 0 \\ \hline
3-5 & Number of slices accepted after L1A for link 1 \\ \hline
... & ...\\ \hline
237-239 & Number of slices accepted after L1A for link 79 \\ \hline
\end{tabular}
\caption{Individually for each input channel it is possible to choose amount of censequtive accepted time slices (max 7, 0 switches off the channel).}
\end{center}
\end{table}

%%%%%
\subsubsection{ROD\_Offsets}
\begin {table}[H]
\begin{center}
\begin{tabular}{|l|l|}
\hline
\textbf{bit} & \textbf{Function} \\
\hline
0-2 & Number of time offset (BC) in relation to L1A starting from which data is taken for link 0 \\ \hline
3-5 & Number of time offset (BC) in relation to L1A starting from which data is taken for link 1 \\ \hline
... & ...\\ \hline
237-239 & Number of time offset (BC) in relation to L1A starting from which data is taken for link 79 \\ \hline
\end{tabular}
\caption{It enables takes data from previous bunch crossings in relation to L1A.}
\end{center}
\end{table}

%%%%%
\subsubsection{ROD\_L1A\_Offset}
\begin {table}[H]
\begin{center}
\begin{tabular}{|l|l|}
\hline
\textbf{bit} & \textbf{Function} \\
\hline
0-7 & Offset of L1A comming from TTCrx to take corresponding BC data (value * BC) \\ \hline
8-31 & reserved \\ \hline
\end{tabular}
\caption{It allows to set addresses of ring memories in a way that corresponding BC is taken (when offset for individual chanel is zero).}
\end{center}
\end{table}

%%%%%
\subsubsection{ROD\_L1A\_Fill}
\begin {table}[H]
\begin{center}
\begin{tabular}{|l|l|}
\hline
\textbf{bit} & \textbf{Function} \\
\hline
0-7 & amount of L1A awaiting for transport \\ \hline
8-31 & reserved \\ \hline
\end{tabular}
\caption{Fill level of L1A FIFO.}
\end{center}
\end{table}

%%%%%
\subsubsection{ROD\_Dbg}
\begin {table}[H]
\begin{center}
\begin{tabular}{|l|l|}
\hline
\textbf{bit} & \textbf{Function} \\
\hline
0-63 & TBD \\ \hline
\end{tabular}
\caption{State machines, internal signals - debug purposes.}
\end{center}
\end{table}


\subsection{I2C\_Error\_Counter}
ErrorCounter for the I2C bus.





\subsection{Test\_RAM: Test RAM}



